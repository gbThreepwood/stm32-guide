\section{Bruk av STM32Cube LL}

\begin{frame}{Introduksjon til low-level biblioteket}
	
	STM32Cube tilbyr to ulike biblitotek for å snakka med dei ulike HW modulane i mikrokontrolleren.
	
	\begin{itemize}
		\item Low Layer (LL)
		\item Hardware Abstraction Layer (HAL)
	\end{itemize}
	
	LL-biblioteket tilbyr eit mindre abstrakt lavnivå API for direkte kommunikasjon med HW.
	
	HAL og LL er heilt uavhengige. Både HAL og LL snakkar direkte med lavnivå I/O-register ved hjelp av peikarar til dei ulike minneadressene.
	
\end{frame}




\begin{frame}[containsverbatim]{STM32Cube LL}
	
\begin{minted}[fontsize=\footnotesize]{c}
#include <stm32f7xx_ll_gpio.h>
#include <stm32f7xx_ll_bus.h>
#include <stm32f7xx_ll_utils.h>

#define USER_LED_PORT           GPIOB
#define USER_LED1_PIN           LL_GPIO_PIN_0
#define USER_LED2_PIN           LL_GPIO_PIN_7
#define USER_LED3_PIN           LL_GPIO_PIN_14

void init_gpio(void) {
	
LL_AHB1_GRP1_EnableClock(LL_AHB1_GRP1_PERIPH_GPIOB);
	
LL_GPIO_SetPinMode(USER_LED_PORT, USER_LED1_PIN,
	LL_GPIO_MODE_OUTPUT);
LL_GPIO_SetPinOutputType(USER_LED_PORT, USER_LED1_PIN,
	LL_GPIO_OUTPUT_PUSHPULL);
}
	\end{minted}
	
\end{frame}

\begin{frame}[containsverbatim]{STM32Cube LL}
	
\begin{minted}[fontsize=\footnotesize]{c}
int main(void) {
	init_gpio();
	
	SystemCoreClockUpdate();
	SysTick_Config(SystemCoreClock / 1000);
	
	while(1){
		
		LL_GPIO_TogglePin(USER_LED_PORT, USER_LED1_PIN);
		LL_mDelay(500);
	}
}
\end{minted}
	
\end{frame}
